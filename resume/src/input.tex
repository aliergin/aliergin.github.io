\documentclass[10pt]{article}

\usepackage{tabularx}
\usepackage{fullpage}
\usepackage{CV}

%%%%%%%%%%%%%%%%%%%%%%%%%%%%%%
 \oddsidemargin  -0.4in
 \evensidemargin -0.4in
 \textwidth      7.1in
 \headheight     0.8in
 \topmargin      -0.5in
 \textheight=8.8in
%%%%%%%%%%%%%%%%%%%%%%%%%%%%%%


\begin{document}

%\pagestyle{empty}
\pagenumbering{Roman}

\begin{center}
%\huge{\textsc{Curriculum Vitae}}

\Large{\textsc{Mesut Ali Ergin}}\\
\vspace{2mm} \normalsize
  239 Montgomery St. Apt. 1E, Highland Park, NJ 08904, USA.\\
Phone: (862)-368-6620 \  Fax: (512)-628-3998\\
%  \hspace{0.5mm}\ \ \ \ \ \ \ \ \ \ 732-445-0614 (Work) \\
  ergin@winlab.rutgers.edu \hspace{5mm} aliergin@gmail.com \\
  http://www.aliergin.com
\end{center}

% \section{Contact Information}
% \begin{flushleft}
%   585 Davis Ave. 2$^{nd}$ Fl. \\
%   Kearny, NJ 07032, USA \\
%   \vspace{2mm}
%   Phone: 862-368-6620 (Mobile)\\
% %  \hspace{0.5mm}\ \ \ \ \ \ \ \ \ \ 732-445-0614 (Work) \\
%   \vspace{2mm}
%   Email: ergin@winlab.rutgers.edu \\
%   \ \ \ \ \ \ \ \ \ \ ergin@ieee.org \\
%   \vspace{2mm}
% %  Homepage: \verb#http://www.winlab.rutgers.edu/~ergin#\\
% \end{flushleft}


%  \section{Personal Details}
%  \begin{flushleft}
%    Gender: Male \\
%    Marital Status: Single \\
%    Date of birth: June 16, 1978 \\
%    Place of birth: Istanbul, TURKEY \\
%    Citizenship: Turkish
%  \end{flushleft}


\section{Objective Statement}
\begin{CV}
\item A full-time research and development position to utilize my prior experience in networking research and
engineering. Available after Feb. 2010 timeframe.
\end{CV}


\section{Education}
\begin{CV}
\item[2004 -- current] Ph.~D. in Electrical and Computer Engineering, Rutgers
University, School of Engineering, New Brunswick, NJ, USA (Expected Graduation:
Feb 2010) \vspace{1mm}\\
Advisors: Prof. Marco Gruteser and Prof. Dipankar Raychaudhuri
\item[2002] M.~S. in Computer Engineering (High Honors), Bogazici University,
Institute for Graduate Studies in Science and Engineering, Istanbul, TURKEY
\vspace{1mm} \\
M.~S. Thesis: \emph{A Cross Layer Protocol for Service Access in Mobile Ad Hoc 
Networks} \\
Advisor: Prof. Cem Ersoy
\item[1999] B.~S. in Control \& Computer Engineering (Honors),
Istanbul Technical University, Electrical and Electronics Engineering
Faculty, Istanbul, TURKEY \vspace{1mm}\\
B.~S. Thesis: \emph{GSM Interface Board Design for a Commercial Payphone} \\
Supervisor: Prof. B. Tevfik Akgun
\item[1995] Istanbul Technical University, Foreign Languages School,
Istanbul, TURKEY
\item[1994] City of London College, English Language School, London, UK
%\item[1994] Kabatas Erkek High School, Istanbul
\end{CV}


\section{Professional Experience}
\begin{CV}
\item[09/2004 -- current] \emph{Graduate Research Assistant}, WINLAB Research
Center, Rutgers University, NJ, USA \\
-- Research Assistantship for ORBIT Wireless Research Testbed under
NSF NRT Grant ANI0335244\\
-- Software development with XML-based technologies\\
-- Various wireless projects under sponsorship research programs for Panasonic,
Intel and Toyota ITC.
\item[05/2006 -- 09/2006] \emph{Graduate Intern}, Intel Corp., Radio Comm. Lab /
CTG, Hillsboro, OR, USA \\
-- Software scheduler development for interference mitigation due to multi-radio
coexistence\\
-- Performance enhancing designs for High Density WLAN environments
\item[06/2005 -- 09/2005] \emph{Graduate Intern}, Intel Corp., Radio Comm. Lab /
CTG, and Wireless Lab / MPG, Santa Clara, CA, USA \\
-- Worked as an engineer in multi-radio coexistence analysis project
for laptop platforms\\
-- Responsible from evaluation of the application-level impacts of co-existence
of different radio technologies like 802.11, UWB, WiMAX etc.
\item[09/1999 -- 08/2004] \emph{Teaching and Research Assistant}, Department of
Computer Engineering, Yeditepe University, Istanbul, TURKEY \\
-- Teaching Assistantship for many undergraduate courses (see Teaching
Experience)\\
-- Research Assistantship for SeMA project under TUBITAK grants
101E037/EEEAG-AY-41 and 101E037/EEEAG-AY-44 \\
-- Senior System Administrator of departmental servers (1000+ users)
\item[06/1998 -- 06/1999] \emph{Contracted Counselor}, Alcatel Inc.,
R\&D Department, Istanbul, TURKEY \\
-- Designed a compact GSM interface system with two processors including
real-time software implementations in both high and low level languages
\item[11/1997 -- 09/2002] \emph{Technical Manager}, A \& A Computers Ltd.,
Istanbul, TURKEY \\
-- Developed network implementations, provided solutions for GNU/Linux systems,
supervised domain maintenance and web design, led software-on-demand
implementations
\end{CV}

\section{Teaching Experience}
\begin{itemize}
\item \emph{Data Communication and Computer Networks}: Developed and maintained
laboratory content for nine semesters\vspace{-2mm}
\item \emph{Special Topics in Computer Networks}: Prepared PS sessions and
organized term projects on high speed and wireless networks topics for two
semesters\vspace{-2mm}
\item \emph{Operating Systems Design}: Taught practical aspects of UNIX-like
OS from the viewpoint of a systems programmer and OS designer for eight
semesters\vspace{-2mm}
\item \emph{Principles of Logic Design}: Conducted hands-on experiments of
basic logic circuitry for one semester.\vspace{-2mm}
\item \emph{Systems Programming and Assembly Language}: Developed and
conducted hands-on experiments on MC6802 based microprocessor kits for one
semester\vspace{-2mm}
\item \emph{Introduction to Computers}: Taught as the first course of CS
juniors for two semesters
\end{itemize}


\section{Research Interests}
\noindent Performance of wireless networks under real deployment
scenarios, involving high density and mobile environments. Creating better
wireless experience for the users of the future Internet. Adaptation approaches
for better multi-radio coexistence.

\section{Publications}
\begin{enumerate}
\item C. Dwertmann, M. A. Ergin, G. Jourjon, M. Ott, T. Rakotoarivelo, I. Seskar, and M. Gruteser, ``DEMO: Mobile Experiments Made Easy with OMF/Orbit'', \emph{In Proceedings of the {ACM SIGCOMM} 2009 (Demo Session)}, Spain, August 2009.

\item G. Chandrasekaran, M. A. Ergin, M. Gruteser, R. P. Martin, J. Yang, and Y. Chen. ``DECODE: Exploiting Shadow Fading to DEtect CO-Moving Wireless DEvices'', {IEEE} Transactions on Mobile Computing (Accepted for Publication), July 2009. 

\item G. Chandrasekaran, M. A. Ergin, J. Yang, S. Liu, Y. Chen, M. Gruteser, and R. P. Martin, ``Empirical Evaluation of the Limits on Localization Using Signal Strength'', \emph{In Proceedings of the 6th IEEE Communications Society Conference on Sensor, Mesh and Ad Hoc Communications and Network (IEEE SECON 2009)}, (Accepted for Publication), Rome, Italy, June 2009.

\item X. Jing, S. Anandaraman, M. A. Ergin, I. Seskar, and
D. Raychaudhuri, ``Distributed Coordination Schemes for Multi-Radio
Co-existence in Dense Spectrum Environments: An Experimental Study on the ORBIT
Testbed'', \emph{In Proceedings of the 3rd IEEE International Symposia on
Dynamic Spectrum Access Networks (IEEE DySPAN 2008)}, pp. 156--166 Chicago, IL,
USA, October 2008.

\item M. A. Ergin, K. Ramachandran, M. Gruteser, ``An Experimental
Study of Inter-cell Interference Effects on System Performance in Unplanned
Wireless LAN Deployments'', \emph{Computer Networks (Elsevier)}, Vol.52,
Issue 14, pp. 2728--2744, October 2008.

\item G. Chandrasekaran, M. A. Ergin, R. P. Martin, M. Gruteser, J. Yang, Y.
Chen, ``DECODE: Detecting Co-Moving Wireless Devices'', \emph{In Proceedings of
the Fifth IEEE International Conference on Mobile Ad-hoc and Sensor Systems
(IEEE MASS 2008)}, pp. 315--320, Atlanta, GA, USA, September 2008.

\item M. A. Ergin, M. Gruteser, L. Luo, D. Raychaudhuri, H. Liu, ``Available
Bandwidth Estimation and Admission Control for QoS Routing in Wireless Mesh
Networks'', \emph{Computer Communications (Elsevier)}, Vol.31, Issue 7, pp.
1301--1317, May 2008.

\item G. Chandrasekaran, M. A. Ergin, M. Gruteser, R. P. Martin, ``Bootstrapping
a Location Service Through Geocoded Postal Addresses'', \emph{In Springer LNCS, 
Proceedings of the 3rd Intl. Symposium on Location and Context-Awareness (LoCA,
held with UbiComp)}, volume 4718, pp. 1--16, Germany, September 2007.

\item M. A. Ergin, K. Ramachandran, M. Gruteser, ``Extended Abstract:
Understanding the Effect of Access Point Density on Wireless LAN Performance'',
\emph{In Proceedings of the ACM International Conference on Mobile Computing
and Networking (ACM MobiCom 2007)}, pp. 350--353, Montreal, Canada, September,
2007.

\item M. A. Ergin, M. Gruteser, ``Using Packet Probes for Available Bandwidth
Estimation: A Wireless Testbed Experience (Poster)'', \emph{In Proceedings of
First ACM SIGMOBILE Int. Workshop on Wireless Network Testbeds, Experimental
Evaluation, and Characterization (WiNTECH 2006, held with Mobicom)}, Los
Angeles, CA, USA, September 2006.

\item S. Baydere, M. A. Ergin, and O. Durmaz, ``Constructing Wireless Sensor
Networks via Effective Topology Maintenance and Querying", \emph{Proc. of
Med-Hoc-Net 2004, 3$^{rd}$ Annual Mediterranean Ad Hoc Networking
Workshop}, Bodrum, Turkey, June 2004

\item M. A. Ergin, ``A Cross Layer Protocol for Service Access in Mobile Ad Hoc
Networks", M.~S. Thesis, \emph{Bogazici University}, 2003

\item E. Ozcan, A. Alkan, S. Demir, M. A. Ergin, H. Kul, and S. E. Seker,
``STARS - Student Transcript, Administration and Registration System, an Open
Source Internet Application", \emph{Fifth National Academic Information
Technologies Conference}, e-paper ref. 87, Turkey, 2003

\item S. Baydere and M. A. Ergin, ``An Architecture for Service Access in Mobile
Ad Hoc Networks", \emph{Proceedings of the IASTED Wireless and Optical
Communications}, pp. 392--397, Banff, Canada, July 2002

\item S. Baydere and M. A. Ergin, ``A Model for Dynamic Service Discovery in
Wireless Ad Hoc Networks", \emph{Proceedings of the Sixth Symposium on
Computer Networks}, pp. 120--128, Gazimagusa, Cyprus, June 2001
\end{enumerate}


\section{Technical Skills}
\begin{CV}
\item[\emph{Programming Skills} ]\ \\
-- Various advanced level programming in C, Java, Pascal, Fortran and Basic
languages\\
-- Systems programing in Linux user and kernel spaces\\
-- Shell programming with BASH and TCL\\
-- Server side web programming with PHP and JSP\\
-- Portal design and web application integration with Apache Cocoon\\
-- Experience in three-tiered programming in PHP, Apache httpd, MySQL environment\\
-- Microprocessor programming in C and assembly (Motorola, Microchip, Atmel families)\\
-- Numerical programming with MATLAB\\

\item[\emph{Administration Skills} ]\ \\
-- Senior level administration of UNIX-based servers, running
Apache, wu-ftpd, OpenSSH, SAMBA, DHCP, NIS, NFS, bind, qmail,
sqwebmail, procmail, iptables, squid, snort, and tripwire\\
-- Database administration of MySQL and Sybase ASA on Linux\\
-- Network administration for various level hardware like layer-3
switches, routers and IEEE 802.11 variant wireless devices

\item[\emph{Other Skills} ]\ \\
-- Extensive advanced use of Linux, Solaris, MS-DOS, and MS Windows 95/98/ME/NT/2000/XP
-- Comprehensive wireless lab experience with spectrum analyzers, signal and traffic generators, scopes, power meters and antennas   
\end{CV}


\section{Professional Memberships}
\begin{itemize}
\item IEEE, Communication Society, Control Society and Computer Society
Member, ACM Member, Turkish Linux Users Association Member
\end{itemize}

\section{Awards}
\begin{itemize}
\item Member of ORBIT Project Team, The Schwarzkopf Award from NSF IUCRC
Association, 2008
\item NSF Travel Award for MobiCom, 2008
\item Fully Supported Graduate Assistant, Rutgers University, 2004-2008
\end{itemize}

\section{Favorites}
\begin{itemize}
\item Latin Dances (Former BUDANS Member), Stage Arts and Polyphonic
Music, Guitar, Sound Systems and Acoustics
\end{itemize}

\section{References}
\begin{table}[h!]
\begin{tabular}{@{}lll@{}}
%\textbf{Prof. Cem Ersoy} \\
%\emph{Thesis Supervisor}\\
%Department of Computer Engineering & Phone: & +90-212-358-1540 (ext.1861)\\
%Bogazici University & Fax: &+90-212-287-2461\\
%Bebek, 34342, Istanbul, TURKEY & Email: & ersoy@boun.edu.tr \\  \vspace{2mm} \\
%\textbf{Prof. Sebnem Baydere} \\
%\emph{Previous Employer}\\
%Department of Computer Engineering & Phone: & +90-216-578-0421\\
%Yeditepe University & Fax: &+90-216-578-0400\\
%Icerenkoy, 34755, Istanbul, TURKEY & Email: & sbaydere@cse.yeditepe.edu.tr \\ 
%\vspace{2mm} \\
\textbf{Prof. Dipankar Raychaudhuri} \\
WINLAB Research Center & Phone: & 732-932-6857 x.638\\
Rutgers University & Fax: & 732-932-6882\\
671 Rt.1 South, North Brunswick, NJ 08902 & 
Email: & ray@winlab.rutgers.edu\\\\
\textbf{Assist. Prof. Marco Gruteser} \\
WINLAB Research Center & Phone: & 732-932-6857 x.649\\
Rutgers University & Fax: & 732-932-6882\\
671 Rt.1 South, North Brunswick, NJ 08902 & 
Email: & gruteser@winlab.rutgers.edu\\\\
\textbf{Dr. Lakshman Krishnamurthy} \\
Intel Corp., CTG & Phone: & 503-264-1624\\
Radio Communication Lab& Fax: & 503-264-3483\\
2111 NE 25th Ave., Hillsboro, OR 97123 & 
Email: & lakshman.krishnamurthy@intel.com\\\\
\textbf{Dr. Camille Chen} \\
Apple Inc. & Phone: & 408-974-7930\\
Wireless System Engineering& Fax: & 408-974-2483\\
1 Infinite Loop, Cupertino, CA 95014 & 
Email: & chen.c@apple.com\\\\

\end{tabular}
\end{table}


% \vspace{2\baselineskip}
% \noindent Last Updated on \today

\end{document}
